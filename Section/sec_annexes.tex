\section*{Annexes}
\label{sec:annexes}
\addcontentsline{toc}{section}{Annexes}
\subsection*{Procédure pour les crédits VEE (SOA et CAS)}
\label{subsec:vee}
\addcontentsline{toc}{subsection}{Procédure pour les crédits VEE (SOA et CAS)}
Une fois que vous avez complété le ou les cours nécessaires pour obtenir votre crédit VEE, vous devez faire la demande pour que celui-ci soit ajouté à votre dossier.\footnote{Vous pouvez trouver des informations concernant les cours qui permettent d'obtenir les crédits VEE dans votre \href{https://www.act.ulaval.ca/programmes-et-cours/premier-cycle/guide-de-letudiant/}{guide étudiant} ou encore sur cette \href{https://soa.org/Education/Exam-Req/Instructions-for-VEE-Directory.aspx}{page de la SOA}.} La procédure est la même pour les candidats de la SOA et les candidats de la CAS. Elle comporte principalement deux étapes: remplir le formulaire de la SOA et acheminer le relevé de notes de l'Université Laval au bureau d'administration des VEE de la SOA.\vspace{\baselineskip} 

\begin{enumerate}
\item \textbf{Remplir le formulaire de la SOA}
\begin{itemize}
\item Se rendre sur la \href{https://soa.org/education/exam-req/edu-vee.aspx}{page d'information sur les VEE}. 
\item Cliquer sur le type d'application pour le VEE qui vous intéresse (\textit{Apply online for VEE Economics Credit}, \textit{Apply online for VEE Corporate Finance Credit} ou \textit{Apply online for VEE Applied Statistics Credit}.
\item Entrer vos informations personnelles et cliquer sur \textit{Select School / University} pour pouvoir accéder à la liste de cours qui sont approuvés pour les VEE.
\item Sélectionner le pays et la province, puis sélectionner l'Université Laval. La liste de cours devrait apparaître. \item Cibler la ligne du cours que vous avez complété et entrer la cote obtenue et la date à laquelle vous avez terminé ce cours. 
\item Cliquer sur \textit{Select} (complètement à gauche de la ligne). Il vous reste ensuite simplement à payer le montant qui sera indiqué.
\end{itemize}\vspace{\baselineskip}
\item \textbf{Acheminer le relevé de notes de l'Université Laval à la SOA}
\begin{itemize}
\item Se rendre sur la \href{https://www.reg.ulaval.ca/cms/DemDoc/releveNotes}{page du Bureau du registraire} pour les relevés de notes.
\item Remplir le formulaire \textbf{REG-910-RN}. Le relevé de notes doit être acheminé au bureau d'administration des VEE de la SOA. L'adresse se trouve sur cette \href{https://soa.org/education/exam-req/course-info/edu-vee-applying-process.aspx}{page}.
\item Acheminer ce formulaire au bureau du registraire en suivant les indications sur la page concernant les relevés de notes.
\end{itemize}
\end{enumerate}


\newpage
\subsection*{Procédure pour l'inscription aux examens préliminaires}
\label{subsec:inscriptionexams}
\addcontentsline{toc}{subsection}{Procédure pour l'inscription aux examens préliminaires}
La procédure pour l'inscription aux examens préliminaires est la même pour les candidats de la CAS et les candidats de la SOA. Elle comporte principalement deux étapes : l'inscription à l'examen sur le site de la SOA et fixer la date sur le site de Prometric.\footnote{Cette dernière étape est omise si vous faites l'examen papier.}\vspace{\baselineskip} 

\begin{enumerate}
\item \textbf{Inscription à l'examen sur le site de la SOA}
\begin{itemize}
\item Se rendre sur la \href{https://www.soa.org/Education/Exam-Req/Registration/edu-registration.aspx}{page d'inscription}. 
\item Sélectionner l'examen de votre choix. Si ce n'est pas déjà fait, vous devrez vous créer un compte sur le site de la SOA.
\item Payer le montant indiqué.
\item Vous recevrez une lettre de confirmation de votre commande par courriel, qui contient notamment votre numéro de candidat.
\end{itemize}\vspace{\baselineskip}

\item \textbf{Fixer la date sur le site de Prometric}
\begin{itemize}
\item Vous recevrez un deuxième courriel (\textit{Letter of confirmation}) qui vous informera de vous inscrire sur le site de Prometric. Ce courriel est envoyé environ 3 à 5 jours ouvrables après avoir payé pour l'examen.
\item Suivre les instructions qui figurent dans ce courriel afin de fixer une date pour votre examen. Pour ce faire, vous devrez utiliser votre numéro d'éligibilité qui figure dans ce second courriel.
\end{itemize}
\end{enumerate}

\newpage
