\section*{Les examens préliminaires}
\label{sec:prelims}
\addcontentsline{toc}{section}{Les examens préliminaires}
Vous trouverez les dates importantes pour les examens préliminaires sur cette \href{https://www.soa.org/Education/Exam-Req/Exam-Day-Info/edu-2018-cbt-test-schedule.aspx}{page}. J'en profite également pour mentionner qu'en vous inscrivant avec votre adresse courriel universitaire, vous serez éligible à un rabais étudiant pour les examens MFE/3F (IFM), C/4 (STAM) et MLC (LTAM). Voir cette \href{https://soa.org/Education/Exam-Req/Syllabus-Study-Materials/Exam-and-Module-Fees.aspx}{page} pour plus de détails concernant les coûts d'inscription.


\subsection*{Examen P/1}
\label{subsec:examp}
\addcontentsline{toc}{subsection}{Examen P/1}
\href{https://www.soa.org/education/exam-req/edu-exam-p-detail.aspx}{L'examen P/1} a pour objectif de développer les notions essentielles de la théorie des probabilités dans le contexte de l'actuariat. La matière de cet examen est traitée dans le cours \textit{ACT-1002 Analyse probabiliste des risques actuariels}. Après avoir complété ce cours, des étudiants ont rapporté avoir étudié \textbf{environ} 50 à 100 heures pour l'examen P/1. Pour consulter les objectifs spécifiques de l'examen ainsi que d'autres informations, voir le \href{https://www.soa.org/Files/Edu/2017/edu-2017-01-p-syllabus.pdf}{syllabus}. Cet examen est d'une durée de 3h00 et comporte 30 questions à choix multiples (choix A, B, C, D, E). La SOA a publié \href{http://www.soa.org/Files/Edu/edu-exam-p-sample-quest.pdf}{326 questions} types pour cet examen, accompagnées du \href{http://www.soa.org/Files/Edu/edu-exam-p-sample-sol.pdf}{solutionnaire}.\vspace{\baselineskip}

À mon avis, le meilleur manuel d'étude pour l'examen P/1 est celui de ACTEX, écrit par Samuel Broverman. Le site \href{http://www.theinfiniteactuary.com/exams/1}{The Infinite Actuary} offre gratuitement quatre examens pratiques qui sont à peu près du même niveau que le véritable examen. C'est donc une excellente pratique pour vous. Enfin, une inscription à \href{https://www.coachingactuaries.com/}{ADAPT} peut également vous permettre de mieux déterminer si vous êtes prêt à faire l'examen et donc s'avérer un investissement très rentable.\vspace{\baselineskip}

Voici un exemple de plan d'étude pour l'examen P/1 (après avoir complété ACT-1002) :
\begin{enumerate}
\item Faire une lecture rapide du manuel ACTEX pour avoir une bonne vue d'ensemble de la matière à l'examen;
\item Relire les sections dans lesquelles vous avez plus de difficulté et faire des exercices jusqu'à devenir confortable;
\item Compléter les problèmes de la SOA en notant les problèmes que vous ne comprenez pas;
\item Relire les sections du manuel ACTEX pour combler les lacunes que vous aviez lorsque vous avez fait les problèmes de la SOA;
\item Débuter les examens pratiques dans le manuel ACTEX;
\item \textbf{Cinq jours avant l'examen :} faire deux examens par jour sur le site The Infinite Actuary;
\item \textbf{Trois jours avant l'examen :} refaire les problèmes que vous ne compreniez pas à l'étape trois et les problèmes ratés dans les examens pratiques;
\item \textbf{La journée avant l'examen :} faire une dernier survol léger de l'ensemble de la matière, relire ses notes et s'assurer de bien connaître les formules nécessaires.
\end{enumerate}
\vspace{\baselineskip}

Ce plan d'étude est seulement à titre indicatif et la majorité d'entre vous auront sans doute besoin de moins d'étude que ce qui y figure. Quelqu'un qui a bien réussi le cours ACT-1002 pourrait très bien faire une lecture rapide du manuel ACTEX, puis passer directement aux examens pratiques. Il faut garder en tête que les problèmes que vous devrez résoudre lors de l'examen P/1 seront moins compliqués que ceux du cours ACT-1002, mais la note de passage de cet examen est d'environ 70\% plutôt que d'être de 50\%. 

\newpage
\subsection*{Examen FM/2}
\label{subsec:examfm}
\addcontentsline{toc}{subsection}{Examen FM/2}
\href{https://www.soa.org/education/exam-req/edu-exam-fm-detail.aspx}{L'examen FM/2} traite des mathématiques financières et des produits dérivés. Les cours du baccalauréat en actuariat en lien avec cet examen sont \textit{ACT-1001 Mathématiques financières} et \textit{ACT-2011 Gestion du risque financier II} pour la partie produits dérivés. Après avoir complété le cours \textit{ACT-1001 Mathématiques financières}, des étudiants ont rapporté avoir étudié \textbf{environ} 75 à 175 heures pour l'examen FM/2. Pour consulter les objectifs spécifiques de l'examen ainsi que d'autres informations, voir le \href{https://www.soa.org/Files/Edu/2017/edu-2017-02-exam-fm-syllabus.pdf}{syllabus}. Cet examen est d'une durée de 3h00 et comporte 35 questions à choix multiples. La SOA a publié \href{https://www.soa.org/Files/Edu/2015/edu-2015-exam-fm-ques-theory.pdf}{133 questions} types sur les mathématiques financières ainsi que le \href{https://www.soa.org/Files/Edu/2015/edu-2015-exam-fm-sol-theory.pdf}{solutionnaire} et \href{https://www.soa.org/Files/Edu/edu-2014-10-exam-fm-ques.pdf}{64 questions} types sur les produits dérivés avec le \href{https://www.soa.org/Files/Edu/edu-2014-10-exam-fm-sol.pdf}{solutionnaire}.\vspace{\baselineskip} 

À mon avis, le meilleur manuel d'étude pour l'examen FM/2 est celui de ASM. Les explications sont extrêmement détaillées en plus d'avoir d'excellentes notes pour bien maîtriser la calculatrice BA II Plus, une compétence qui est très utile lors de l'examen. Personnellement, je n'avais pas utilisé ADAPT pour cet examen, puisque le manuel comporte déjà énormément de problèmes en plus d'examens pratiques qui sont comparables en difficulté au véritable examen. Vous trouverez, encore une fois, deux examens pratiques tout à fait gratuits sur le site \href{http://www.theinfiniteactuary.com/exams/2}{The Infinite Actuary}. La difficulté de ces deux examens pratiques est comparable au véritable examen.\vspace{\baselineskip} 

\textbf{Changements entrant en vigueur en octobre 2018:} Le nouvel examen ressemblera beaucoup à celui en vigueur depuis juin 2017 à l'exception de certaines notions du VEE \textit{Corporate Finance} qui seront maintenant évaluées. Ainsi, l'essentiel de la matière qui se trouve à l'examen sera traitée dans les cours \textit{ACT-1001 Mathématiques financières} et \textit{ACT-1006 Gestion du risque financier I}.


\newpage
\subsection*{Examen IFM/3F (Anciennement MFE)}
\label{subsec:exammfe}
\addcontentsline{toc}{subsection}{Examen IFM/3F}
\href{https://www.soa.org/education/exam-req/edu-exam-mfe-detail.aspx}{L'examen IFM/3F} traite des modèles financiers et de leurs applications dans des contextes d'assurance et de gestion de risques. Les notions de produits dérivés, vues dans l'examen FM/2, seront utilisées comme base. Le cour du baccalauréat en lien avec cet examen est \textit{ACT-2011 Gestion du risque financier II}. Pour consulter les objectifs spécifiques de l'examen ainsi que d'autres informations, voir le \href{https://www.soa.org/Files/Edu/2016/edu-2016-11-mfe-syllabus.pdf}{syllabus}. Cet examen est d'une durée de 3h00 et comporte 30 questions à choix multiples. La SOA a publié \href{http://www.soa.org/files/edu/edu-exam-mfe-sample-quest-sol.pdf}{76 questions} types pour cet examen. Je vous conseille toutefois de ne pas trop perdre de temps sur ces questions, puisqu'elles ne sont pas représentatives de l'examen.\vspace{\baselineskip}

Pour ce qui est du manuel d'étude à utiliser, le plus populaire est celui de ASM, écrit par Abraham Weishaus. Toutefois, le \href{http://howardmahler.com/Teaching/MFE.html}{manuel} d'Howard Mahler est beaucoup apprécié par plusieurs et coûte seulement 50 USD en version électronique. Par contre, la principale critique au sujet de ce manuel est qu'il y a des lacunes au niveau des concepts plus avancés (mouvement brownien, lemme d'Itô, modèles pour les taux d'intérêt, etc.)\vspace{\baselineskip}

Pour cet examen, ADAPT est un outil extrêmement utile. Plusieurs personnes sur des forums de discussion disent que c'est probablement l'examen où les questions dans ADAPT sont le plus similaires aux questions lors du véritable examen. D'ailleurs, beaucoup d'étudiants rapportent qu'ils ont rencontré des questions dans l'examen formulées exactement de la même façon que des questions dans ADAPT. De plus, les examens pratiques du livre ASM ne sont pas tout à fait représentatifs du véritable examen et c'est pourquoi je recommande très fortement l'utilisation de ADAPT pour l'examen MFE/3F.\vspace{\baselineskip}

\newpage

\section*{Les examens préliminaires (partie II)}
\label{sec:prelimssuite}
\addcontentsline{toc}{section}{Les examens préliminaires (partie II) }
\subsection*{Examen STAM (SOA) (Anciennement C)}
\label{subsec:examSTAM}
\addcontentsline{toc}{subsection}{Examen STAM (SOA)}
\href{https://www.soa.org/education/exam-req/edu-exam-c-detail.aspx}{L'examen STAM} est une introduction à la modélisation, à l'analyse de données statistiques dans un contexte d'affaires, à l'estimation et à la création d'intervalles de confiance pour les décisions basées sur les modèles statistiques. Les cours du baccalauréat en lien avec cet examen sont \textit{ACT-2001 Introduction à l'actuariat II}, \textit{ACT-2005 Mathématiques actuarielles IARD I} et \textit{ACT-2008 Mathématiques actuarielles IARD II}. Pour consulter les objectifs spécifiques de l'examen ainsi que d'autres informations, voir le \href{https://www.soa.org/Files/Edu/2017/edu-2017-02-exam-c-syllabus.pdf}{syllabus}. Cet examen est d'une durée de 3h30 et comporte 35 questions à choix multiples. La SOA a publié \href{http://www.soa.org/files/edu/edu-exam-c-sample-quest.pdf}{305 questions} types pour cet examen, accompagnées du \href{http://www.soa.org/files/edu/edu-exam-c-sample-sol.pdf}{solutionnaire}.\vspace{\baselineskip}

Pour ce qui est du manuel d'étude à utiliser, il y a principalement deux écoles de pensée qui s'affrontent. Le manuel le plus populaire est probablement celui de ASM, encore une fois écrit par Abraham Weishaus. Par contre, certaines personnes estiment que les explications sont souvent trop peu détaillées et qu'il est donc difficile de bien comprendre la matière avec ce manuel. À l'opposé, les gens qui utilisent le \href{http://howardmahler.com/Teaching/C.html}{manuel} d'Howard Mahler disent que l'auteur prend le temps de bien expliquer toute la théorie et qu'il présente un bon nombre d'astuces qui aident à gagner en rapidité au moment de l'examen. Le seul problème, c'est que le manuel de Mahler fait 5528 pages, contre 1282 pages pour le livre ASM (en excluant les examens pratiques --- il n'y a pas d'examens pratiques dans le livre de Mahler, ils doivent être achetés séparément). \vspace{\baselineskip}

Malgré tout, pour ceux qui ont déjà suivi les cours du baccalauréat qui traitent de la matière à l'examen C/4, le manuel ASM est sans doute suffisamment détaillé pour que vous atteigniez un niveau de préparation adéquat. Pour ceux qui voudraient faire l'examen avant de suivre les cours de mathématiques actuarielles IARD, je conseille d'utiliser le livre de Mahler.\vspace{\baselineskip}

Pour ce qui est des examens pratiques, plusieurs personnes sur les forums de discussion ont de très bons commentaires pour ceux d'\href{http://howardmahler.com/Teaching/C.html}{Howard Mahler}, vendus séparément de son livre. Bien qu'ils semblent généralement être considérés comme difficiles, ces examens pratiques permettent vraiment de bien appliquer la matière et d'établir des connexions entre les différents sujets. À l'opposé, plusieurs ont tendance à dire que ADAPT n'est pas un bon outil pour cet examen, puisque la formulation des questions n'est pas représentative du véritable examen. 

\subsection*{Examen MAS-2 (CAS)(Anciennement C)}
\label{subsec:examSTAM}
\addcontentsline{toc}{subsection}{Examen MAS-2 (CAS)}
\href{http://www.casact.org/admissions/syllabus/index.cfm?fa=MASII&parentID=392}{L'examen MAS-2} est une introduction à la modélisation, à l'analyse de données statistiques dans un contexte d'affaires, à la crédibilité, à l'estimation et à la création d'intervalles de confiance pour les décisions basées sur les modèles statistiques. Les cours du baccalauréat en lien avec cet examen sont \textit{ACT-2001 Introduction à l'actuariat II}, \textit{ACT-2005 Mathématiques actuarielles IARD I} et \textit{ACT-2008 Mathématiques actuarielles IARD II}. Pour consulter les objectifs spécifiques de l'examen ainsi que d'autres informations, voir le \href{http://www.casact.org/cms/files/Exam_MAS-II_2018_v20_2017_01_13_1.pdf}{syllabus}. Cet examen est d'une durée de 4h00 à correction négative et le nombre de questions n'est pas encore annoncé. La SOA a publié \href{http://www.soa.org/files/edu/edu-exam-c-sample-quest.pdf}{305 questions} types pour cet examen, accompagnées du \href{http://www.soa.org/files/edu/edu-exam-c-sample-sol.pdf}{solutionnaire}. Il s'agit du matériel d'étude du C, du nouveau matériel d'étude sera disponible d'ici là.\vspace{\baselineskip}

Pour ce qui est du manuel d'étude et des examens formatifs à utiliser, le matériel devrait être similaire à l'ancien C. Soit le manuel de ASM, encore une fois écrit par Abraham Weishaus et le \href{http://howardmahler.com/Teaching/C.html}{manuel} d'Howard Mahler. Plus d'information sera ajoutée avec les développements des changements.\vspace{\baselineskip}


%\subsection*{Examen MLC (SOA)}
%\label{subsec:exammlc}
%\addcontentsline{toc}{subsection}{Examen MLC (SOA)}



\subsection*{Examen MAS-1 (CAS) (Anciennement S)}\label{subsec:examMAS-1}
\addcontentsline{toc}{subsection}{Examen MAS-1 (CAS)}

\href{http://www.casact.org/admissions/syllabus/index.cfm?fa=MASI&parentID=391}{L'examen STAM/MAS-1} est une introduction aux modèles probabilistes (les processus stochastiques et les modèles de survie), à l'utilisation des tests statistiques, aux modèles linéaires et aux séries temporelles. Les cours du baccalauréat en lien avec cet examen sont \textit{ACT-2009 Processus stochastiques}, \textit{ACT-2003 Modèles linéaires en actuariat} et \textit{ACT-2010 Séries chronologiques}. Pour consulter les objectifs spécifiques de l'examen ainsi que d'autres informations, voir le \href{http://www.casact.org/cms/files/Exam_MAS-I_2018_v20_2017_01_13_2.pdf}{syllabus}. Cet examen est d'une durée de 4h et devrait comporter 40 questions à choix multiples à correction négative comme l'examen S. La CAS a publié des anciens \href{http://www.casact.org/admissions/studytools/examS/}{examens} types du S, ainsi que quelques \href{http://www.casact.org/admissions/studytools/examS/Sample_Questions.pdf}{questions} de l'examen S.\vspace{\baselineskip}

Pour ce qui est du manuel d'étude à utiliser, il y a le manuel d'\href{https://drive.google.com/open?id=0B6kXivc6X9LISGhkVUkzLW5sSnc}{ASM}, encore une fois écrit par Abraham Weishaus. Par contre, certaines personnes estiment que les \href{https://drive.google.com/open?id=0B6kXivc6X9LIOUs3SDF3NmVKNGM}{examens pratiques} d'Howard Mahler sont très similaire aux examens de facto plus adéquats pour l'étude. Par contre, avec les changements la situation pourrait changer. \vspace{\baselineskip}
