\section*{Annexes}
\label{sec:annexes}
\addcontentsline{toc}{section}{Annexes}

\subsection*{Procédure pour l'inscription aux examens préliminaires}
\label{subsec:inscriptionexams}
\addcontentsline{toc}{subsection}{Procédure pour l'inscription aux examens préliminaires}
La procédure pour l'inscription aux examens préliminaires est la même pour les candidats de la CAS et les candidats de la SOA. Elle comporte principalement deux étapes : l'inscription à l'examen sur le site de la SOA et fixer la date sur le site de Prometric.\footnote{Cette dernière étape est omise si vous faites l'examen papier.}\vspace{\baselineskip} 

\begin{enumerate}
\item \textbf{Inscription à l'examen sur le site de la SOA}
\begin{itemize}
\item Se rendre sur la \href{https://www.soa.org/Education/Exam-Req/Registration/edu-registration.aspx}{page d'inscription}. 
\item Sélectionner l'examen de votre choix. Si ce n'est pas déjà fait, vous devrez vous créer un compte sur le site de la SOA.
\item Payer le montant indiqué.
\item Vous recevrez une lettre de confirmation de votre commande par courriel, qui contient notamment votre numéro de candidat.
\end{itemize}\vspace{\baselineskip}

\item \textbf{Fixer la date sur le site de Prometric}
\begin{itemize}
\item Vous recevrez un deuxième courriel (\textit{Letter of confirmation}) qui vous informera de vous inscrire sur le site de Prometric. Ce courriel est envoyé environ 3 à 5 jours ouvrables après avoir payé pour l'examen.
\item Suivre les instructions qui figurent dans ce courriel afin de fixer une date pour votre examen. Pour ce faire, vous devrez utiliser votre numéro d'éligibilité qui figure dans ce second courriel.
\end{itemize}
\end{enumerate}

\newpage


\subsection*{Procédure pour réclamer les crédits VEE (SOA et CAS)}
\label{subsec:reclamvee}
\addcontentsline{toc}{subsection}{Procédure pour réclamer les crédits VEE (SOA et CAS)}
Une fois que vous avez complété le ou les cours nécessaires pour obtenir votre crédit VEE, vous devez faire la demande pour que celui-ci soit ajouté à votre dossier.\footnote{Vous pouvez trouver des informations concernant les cours et les cotes qui permettent d'obtenir les crédits VEE dans votre \href{https://www.act.ulaval.ca/programmes-et-cours/premier-cycle/guide-de-letudiant/}{guide étudiant} ou encore sur cette \href{https://soa.org/Education/Exam-Req/Instructions-for-VEE-Directory.aspx}{page de la SOA}.} Un autre point important à savoir est qu'il faut avoir réussi au moins 2 examens professionnels avant de pouvoir réclamer des VEE. Il n'y a aucune limite de temps pour réclamer des VEE: si le cours réussi correspond aux exigences de la SOA au moment où vous réussi le cours \textbf{ou} en ce moment, vous pouvez réclamer le VEE correspondant, peu importe à quel moment le cours a été réussi.\vspace{\baselineskip} 

La procédure est la même pour les candidats de la SOA et les candidats de la CAS. Elle comporte principalement deux étapes: remplir le formulaire en ligne de la SOA et acheminer le relevé de notes de l'Université Laval au bureau d'administration des VEE de la SOA.\vspace{\baselineskip} 

\begin{enumerate}
\item \textbf{Remplir le formulaire de la SOA}
\begin{itemize}
\item Se rendre sur la \href{https://soa.org/education/exam-req/edu-vee.aspx}{page d'information sur les VEE}. 
\item Cliquer sur le type d'application en ligne pour le VEE qui vous intéresse (\textit{Apply for VEE Economics Credit}, \textit{Apply for VEE Accounting and Finance Credit} ou \textit{Apply for VEE Mathematical Statistics Credit}).
\item Entrer vos informations personnelles et cliquer sur \textit{Select School / University} pour pouvoir accéder à la liste de cours qui sont approuvés pour les VEE.
\item Sélectionner le pays et la province, puis sélectionner l'Université Laval. La liste de cours devrait apparaître. 
\item Cibler la ligne du cours que vous avez complété et entrer la cote obtenue et la date à laquelle vous avez terminé ce cours. 
\item Cliquer sur \textit{Select} (complètement à gauche de la ligne). Il vous reste ensuite simplement à payer le montant qui sera indiqué. En ce moment, le montant est de 50\$ US par VEE réclamé.
\end{itemize}\vspace{\baselineskip} \newpage
\item \textbf{Acheminer le relevé de notes de l'Université Laval à la SOA}
\begin{itemize}
\item Se rendre sur la \href{https://www.reg.ulaval.ca/cms/DemDoc/releveNotes}{page du Bureau du registraire pour les relevés de notes}.
\item Remplir le formulaire \textbf{REG-910-RN}. Le relevé de notes doit être acheminé au bureau d'administration des VEE de la SOA. L'adresse se trouve sur cette \href{https://soa.org/education/exam-req/course-info/edu-vee-applying-process.aspx}{page}.
\item Acheminer ce formulaire au bureau du registraire en suivant les indications sur la page concernant les relevés de notes. Des frais de 9\$ sont à prévoir pour la production du relevé de notes officiel.
\end{itemize}
\end{enumerate}

\newpage

\subsection*{Procédure pour réclamer une accréditation d'examen professionnel de l'ICA}
\label{subsec:reclamaccredit}
\addcontentsline{toc}{subsection}{Procédure pour réclamer une accréditation d'examen professionnel de l'ICA}

La procédure pour réclamer des accréditations de l'ICA est similaire sur certains points à la procédure pour réclamer des crédits VEE. Si vous avez eu les notes suffisantes dans certains cours pour pouvoir réclamer des accréditations\footnote{Vous pouvez trouver des informations concernant les cours et les cotes qui permettent d'obtenir les accréditations dans votre \href{https://www.act.ulaval.ca/programmes-et-cours/premier-cycle/guide-de-letudiant/}{guide étudiant}.}, vous avez trois ans pour suivre la procédure suivante, qui comporte principalement deux étapes: remplir le formulaire de l'ICA et l'acheminer au bureau de l'ICA avec un relevé de notes de l'Université Laval.\vspace{\baselineskip} 

\begin{enumerate}
\item \textbf{Remplir le formulaire de l'ICA}
\begin{itemize}
\item Se rendre sur la \href{http://www.cia-ica.ca/fr/adhesion/programme-d-agr\%C3\%A9ment-universitaire-(pau)-de-l-ica---page-d'accueil/renseignements-pour-les-candidats}{page d'information sur les accréditations d'examens}. 
\item Imprimer et remplir à la main le formulaire de demande disponible sur la page.
\item En ce moment, les frais demandés sont de 75\$ CAD par accréditation réclamée. Vous pouvez régler ces frais en ligne, par téléphone ou en joignant un chèque à votre formulaire; les informations pertinentes à ce sujet sont exposées dans le formulaire.
\end{itemize}\vspace{\baselineskip}
\item \textbf{Acheminer le formulaire et un relevé de notes officiel à l'ICA}
\begin{itemize}
\item Le but est d'avoir en votre possession votre relevé de notes officiel de l'université Laval et de le joindre vous-même au formulaire de l'ICA. Pour ce faire, vous pouvez simplement vous présenter au Bureau du registraire et demander la production d'un relevé de notes officiel (et par le fait même, ne pas avoir à remplir de formulaire).
\item Si vous ne pouvez pas vous rendre physiquement au Bureau du registraire, se rendre sur la \href{https://www.reg.ulaval.ca/cms/DemDoc/releveNotes}{page du Bureau du registraire pour les relevés de notes} et remplir le formulaire \textbf{REG-910-RN}. Désignez-vous comme destinataire, ce qui vous permettra de recevoir votre relevé chez vous. Acheminer ce formulaire au bureau du registraire en suivant les indications sur la page concernant les relevés de notes. 
\item Des frais de 9\$ sont à prévoir pour la production du relevé de notes officiel, que ça soit sur place ou avec l'envoi d'un formulaire.
\item Maintenant que vous avez le relevé de notes officiel en votre possession, il est très important de ne pas ouvrir l'enveloppe, ce qui briserait le sceau officiel du registraire et causerait le refus de votre demande.
\item Acheminer le formulaire de l'ICA, le relevé de notes officiel et le chèque (si vous payez par chèque) au bureau de l'ICA à Ottawa. L'adresse se trouve sur le formulaire de demande de l'ICA.
\end{itemize}
\end{enumerate}

\newpage

