\section*{Les examens préliminaires (partie I) - Communs SOA et CAS}
\label{sec:prelims}
\addcontentsline{toc}{section}{Les examens préliminaires (partie I) - Communs SOA et CAS}

\subsection*{Examen P/1}
\label{subsec:examp}
\addcontentsline{toc}{subsection}{Examen P/1}
\href{https://www.soa.org/education/exam-req/edu-exam-p-detail.aspx}{L'examen P/1} (P pour \textit{Probability}) a pour objectif de développer les notions essentielles de la théorie des probabilités dans le contexte de l'actuariat. La matière de cet examen est traitée dans le cours \textit{ACT-1002 Analyse probabiliste des risques actuariels}. Après avoir complété ce cours, des étudiants ont rapporté avoir étudié \textbf{environ} 50 à 100 heures pour l'examen P/1. Pour consulter les objectifs spécifiques de l'examen ainsi que d'autres informations, voir le syllabus. Cet examen est d'une durée de 3 heures et comporte 30 questions à choix multiples (choix A, B, C, D, E). La SOA a publié \href{http://www.soa.org/Files/Edu/edu-exam-p-sample-quest.pdf}{326 questions} types pour cet examen, accompagnées du \href{http://www.soa.org/Files/Edu/edu-exam-p-sample-sol.pdf}{solutionnaire}.\vspace{\baselineskip}

À mon avis, le meilleur manuel d'étude pour l'examen P/1 est celui de ASM, écrit par Abraham Weishaus. Le site \href{http://www.theinfiniteactuary.com/exams/1}{The Infinite Actuary} offre gratuitement quatre examens pratiques qui sont à peu près du même niveau que le véritable examen. C'est donc une excellente pratique pour vous. Enfin, une inscription à \href{https://www.coachingactuaries.com/}{ADAPT} peut également vous permettre de mieux déterminer si vous êtes prêt à faire l'examen et donc s'avérer être un investissement très rentable.\vspace{\baselineskip}

Voici un exemple de plan d'étude pour l'examen P/1 (après avoir complété ACT-1002) :
\begin{enumerate}
\item Faire une lecture rapide du manuel ASM pour avoir une bonne vue d'ensemble de la matière à l'examen;
\item Relire les sections dans lesquelles vous avez plus de difficulté et faire des exercices jusqu'à devenir confortable;
\item Compléter les problèmes de la SOA en notant les problèmes que vous ne comprenez pas;
\item Relire les sections du manuel ASM pour combler les lacunes que vous aviez lorsque vous avez fait les problèmes de la SOA;
\item Débuter les examens pratiques dans le manuel ASM;
\item \textbf{Cinq jours avant l'examen :} faire deux examens par jour sur le site The Infinite Actuary;
\item \textbf{Trois jours avant l'examen :} refaire les problèmes que vous ne compreniez pas à l'étape trois et les problèmes ratés dans les examens pratiques;
\item \textbf{La journée avant l'examen :} faire une dernier survol léger de l'ensemble de la matière, relire ses notes et s'assurer de bien connaître les formules nécessaires.
\end{enumerate}
\vspace{\baselineskip}

Ce plan d'étude est seulement à titre indicatif et la majorité d'entre vous auront sans doute besoin de moins d'étude que ce qui y figure. Quelqu'un qui a bien réussi le cours ACT-1002 pourrait très bien faire une lecture rapide du manuel ASM, puis passer directement aux examens pratiques. Il faut garder en tête que les problèmes que vous devrez résoudre lors de l'examen P/1 seront moins compliqués que ceux du cours ACT-1002, mais la note de passage de cet examen est d'environ 70\% plutôt que d'être de 50\%. 

\newpage
\subsection*{Examen FM/2}
\label{subsec:examfm}
\addcontentsline{toc}{subsection}{Examen FM/2}
\href{https://www.soa.org/education/exam-req/edu-exam-fm-detail.aspx}{L'examen FM/2} (FM pour \textit{Financial Mathematics}) traite surtout des mathématiques financières et un peu des produits dérivés. Le cours du baccalauréat en actuariat en lien avec cet examen est \textit{ACT-1001 Mathématiques financières}, pour la plus grande partie, mais quelques notions supplémentaires de \textit{ACT-1006 Gestion du risque financier I} sont traitées. Après avoir complété le cours \textit{ACT-1001 Mathématiques financières}, des étudiants ont rapporté avoir étudié \textbf{environ} 75 à 175 heures pour l'examen FM/2. Pour consulter les objectifs spécifiques de l'examen ainsi que d'autres informations, voir le syllabus. Cet examen est d'une durée de 3 heures et comporte 35 questions à choix multiples. La SOA a publié \href{http://www.soa.org/Files/Edu/2017/exam-fm-sample-questions.pdf}{202 questions} types pour l'examen ainsi que leur \href{http://www.soa.org/Files/Edu/2017/exam-fm-sample-solutions.pdf}{solutionnaire}.\vspace{\baselineskip} 

À mon avis, le meilleur manuel d'étude pour l'examen FM/2 est celui de ASM. Les explications sont extrêmement détaillées en plus d'avoir d'excellentes notes pour bien maîtriser la calculatrice BA II Plus, une compétence qui est très utile lors de l'examen. Personnellement, je n'avais pas utilisé ADAPT pour cet examen, puisque le manuel comporte déjà énormément de problèmes en plus d'examens pratiques qui sont comparables en difficulté au véritable examen. Vous trouverez, encore une fois, deux examens pratiques tout à fait gratuits sur le site \href{http://www.theinfiniteactuary.com/exams/2}{The Infinite Actuary}. La difficulté de ces deux examens pratiques est comparable au véritable examen.\vspace{\baselineskip} 

\textbf{Changements du curriculum de juillet 2018:} Les changements apportés à l'examen FM ont été amenés en deux temps. L'examen a d'abord changé en juin 2017: en effet, depuis cette date, les seules notions de produits dérivés évaluées pour l'examen FM sont les swaps de taux d'intérêt. Un autre changement de juin 2017 est l'arrivée de quelques notions qui étaient auparavant dans le VEE \textit{Corporate Finance} et de quelques notions qualitatives sur les facteurs qui peuvent influencer les taux d'intérêt. Le deuxième changement de l'examen est prévu en octobre 2018. Les changements de ce mois sont assez minimes: en gros, les fonds d'amortissement sont enlevés du syllabus, et les pondérations des différents sujets changent légèrement. Je vous conseille de consulter le syllabus pour plus d'informations sur les notions traitées à l'examen.


\newpage
\subsection*{Examen IFM/3F (Anciennement MFE)}
\label{subsec:exammfe}
\addcontentsline{toc}{subsection}{Examen IFM/3F (Anciennement MFE)}
\href{https://www.soa.org/Education/Exam-Req/edu-exam-ifm-detail.aspx}{L'examen IFM/3F} (IFM pour \textit{Investment and Financial Markets}) traite de la base théorique de la finance corporative et des modèles financiers et de leurs applications dans des contextes d'assurance et de gestion de risques. Les notions de produits dérivés seront utilisées comme base. Le cours du baccalauréat en lien avec cet examen est \textit{ACT-2011 Gestion du risque financier II}. Pour consulter les objectifs spécifiques de l'examen ainsi que d'autres informations, voir le syllabus. Cet examen est d'une durée de 3 heures et comporte 30 questions à choix multiples. La SOA a publié \href{http://www.soa.org/files/edu/edu-exam-mfe-sample-quest-sol.pdf}{76 questions} types pour cet examen. Je vous conseille toutefois de ne pas trop perdre de temps sur ces questions, puisqu'elles ne sont pas représentatives de l'examen.\vspace{\baselineskip}

Pour ce qui est du manuel d'étude à utiliser, le plus populaire est celui de ASM, écrit par Abraham Weishaus. Toutefois, le \href{http://howardmahler.com/Teaching/MFE.html}{manuel} d'Howard Mahler est beaucoup apprécié par plusieurs et coûte seulement 50 USD en version électronique. Par contre, la principale critique au sujet de ce manuel est qu'il y a des lacunes au niveau des concepts plus avancés (mouvement brownien, lemme d'Itô, modèles pour les taux d'intérêt, etc.)\vspace{\baselineskip}

Pour cet examen, ADAPT est un outil extrêmement utile. Plusieurs personnes sur des forums de discussion disent que c'est probablement l'examen où les questions dans ADAPT sont le plus similaires aux questions lors du véritable examen. D'ailleurs, beaucoup d'étudiants rapportent qu'ils ont rencontré des questions dans l'examen formulées exactement de la même façon que des questions dans ADAPT. De plus, les examens pratiques du livre ASM ne sont pas tout à fait représentatifs du véritable examen et c'est pourquoi je recommande très fortement l'utilisation de ADAPT pour l'examen IFM/3F.\vspace{\baselineskip}

\textbf{Changements du curriculum de juillet 2018:} Comme c'était déjà le cas depuis juin 2017, les concepts plus poussés mathématiquement (mouvement brownien, processus stochastiques, modèles de taux d'intérêt) sont retirés du syllabus de cet examen et seront plutôt traités dans les examens avancés de Fellowship appropriés. Également, à partir de juillet 2018, certaines notions de finance corporative et de théorie moderne du portefeuille sont ajoutés au syllabus de l'examen IFM. Je vous conseille de consulter le syllabus pour plus d'informations sur les notions traitées à l'examen.


\newpage




\section*{Les examens préliminaires (partie II) - Côté SOA}
\label{sec:prelimssuitesoa}
\addcontentsline{toc}{section}{Les examens préliminaires (partie II) - Côté SOA}

\subsection*{Examen STAM (Anciennement C)}
\label{subsec:examstam}
\addcontentsline{toc}{subsection}{Examen STAM (Anciennement C)}
\href{https://www.soa.org/Education/Exam-Req/edu-exam-stam-detail.aspx}{L'examen STAM} (STAM pour \textit{Short-Term Actuarial Mathematics}) est une introduction à la modélisation, à l'analyse de données statistiques dans un contexte d'affaires, à l'estimation et à la création d'intervalles de confiance pour les décisions basées sur les modèles statistiques. Les cours du baccalauréat en lien avec cet examen sont \textit{ACT-2001 Introduction à l'actuariat II}, \textit{ACT-2005 Mathématiques actuarielles IARD I} et \textit{ACT-2008 Mathématiques actuarielles IARD II}. Pour consulter les objectifs spécifiques de l'examen ainsi que d'autres informations, voir le syllabus. Cet examen est d'une durée de 3h30 et comporte 35 questions à choix multiples. La SOA a publié \href{http://www.soa.org/files/edu/edu-exam-c-sample-quest.pdf}{305 questions} types pour cet examen, accompagnées du \href{http://www.soa.org/files/edu/edu-exam-c-sample-sol.pdf}{solutionnaire}.\vspace{\baselineskip}

Pour ce qui est du manuel d'étude à utiliser, il y a principalement deux écoles de pensée qui s'affrontent. Le manuel le plus populaire est probablement celui de ASM, encore une fois écrit par Abraham Weishaus. Par contre, certaines personnes estiment que les explications sont souvent trop peu détaillées et qu'il est donc difficile de bien comprendre la matière avec ce manuel. À l'opposé, les gens qui utilisent le \href{http://howardmahler.com/Teaching/C.html}{manuel} d'Howard Mahler disent que l'auteur prend le temps de bien expliquer toute la théorie et qu'il présente un bon nombre d'astuces qui aident à gagner en rapidité au moment de l'examen. Le seul problème, c'est que le manuel de Mahler fait 5528 pages, contre 1282 pages pour le livre ASM (en excluant les examens pratiques --- il n'y a pas d'examens pratiques dans le livre de Mahler, ils doivent être achetés séparément). \vspace{\baselineskip}

Malgré tout, pour ceux qui ont déjà suivi les cours du baccalauréat qui traitent de la matière à l'examen STAM/4, le manuel ASM est sans doute suffisamment détaillé pour que vous atteigniez un niveau de préparation adéquat. Pour ceux qui voudraient faire l'examen avant de suivre les cours de mathématiques actuarielles IARD, je conseille d'utiliser le livre de Mahler.\vspace{\baselineskip}

Pour ce qui est des examens pratiques, plusieurs personnes sur les forums de discussion ont de très bons commentaires pour ceux d'\href{http://howardmahler.com/Teaching/C.html}{Howard Mahler}, vendus séparément de son livre. Bien qu'ils semblent généralement être considérés comme difficiles, ces examens pratiques permettent vraiment de bien appliquer la matière et d'établir des connexions entre les différents sujets. À l'opposé, plusieurs ont tendance à dire que ADAPT n'est pas un bon outil pour cet examen, puisque la formulation des questions n'est pas représentative du véritable examen. \vspace{\baselineskip}

\textbf{Changements du curriculum de juillet 2018:} Parmi tous les examens déjà existants, l'ancien examen C est celui qui a subi le plus gros changement avec les nouvelles exigences de juillet 2018. Le matériel statistique de base qui était auparavant dans le syllabus de l'examen C est maintenant évalué dans le VEE \textit{Mathematical Statistics}. Par conséquent, les connaissances de ce VEE sont maintenant considérées comme prérequises à l'examen, bien qu'elles ne soient pas évaluées dans ce dernier. De plus, le matériel traitant de tables de mortalités est maintenant évalué dans l'exam LTAM. Ces sujets sont remplacés par des notions de produits d'assurance court terme (entre autres, assurance santé, assurance habitation et assurance responsabilité). Des notions de tarification et de méthodes de réserves pour les contrats d'assurance court terme sont égalements ajoutées au syllabus. Je vous conseille de consulter ce dernier pour plus d'informations sur les notions traitées à l'examen. \vspace{\baselineskip}

\newpage


\subsection*{Examen LTAM (Anciennement MLC)}
\label{subsec:examltam}
\addcontentsline{toc}{subsection}{Examen LTAM (Anciennement MLC)}

\href{https://www.soa.org/Education/Exam-Req/edu-exam-ltam-detail.aspx}{L'examen LTAM} (LTAM pour \textit{Long-Term Actuarial Mathematics}) évalue les connaissances théoriques sur les modèles de paiements contingents (des engagements à long terme auxquels sont associés des flux monétaires qui peuvent se produire ou non), de même que l'application de ces modèles à l'assurance et à divers risques financiers.  Les cours du baccalauréat en lien avec cet examen sont \textit{ACT-2004 Mathématiques actuarielles vie I}, \textit{ACT-2007 Mathématiques actuarielles vie II}, \textit{ACT-4106 Modèles avancés en assurance de personnes} et \textit{ACT-4101 Régimes de retraite}. Pour consulter les objectifs spécifiques de l'examen ainsi que d'autres informations, voir le syllabus. Cet examen est d'une durée de 4 heures et le nombre de questions n'est pas encore annoncé. \vspace{\baselineskip}

Contrairement aux autres examens préliminaires de la SOA, l'examen LTAM ne peut pas être fait dans un centre Prometric; on doit absolument le faire en version papier. Cette différence est expliquée par le fait que l'examen LTAM est le seul examen préliminaire à contenir non seulement des questions à choix multiples, mais aussi des questions à développement, ce qui le rapproche d'une certaine manière aux examens avancés. Une autre différence notable est que l'examen LTAM est le seul examen préliminaire auquel on peut avoir accès aux examens des sessions précédentes. Vous pouvez les trouver sur \href{https://www.soa.org/education/exam-req/syllabus-study-materials/edu-multiple-choice-exam.aspx}{cette page} (cherchez les examens MLC, qui correspondent à l'ancien nom de l'examen LTAM). Il va donc de soit que les examens précédents sont une ressource essentielle pour ceux qui comptent se préparer à passer cet examen. En plus de ces examens précédents, la SOA a aussi publié \href{http://www.soa.org/files/edu/edu-2014-spring-mlc-ques.pdf}{322 questions types à choix multiples} (accompagnées de leur \href{http://www.soa.org/files/edu/edu-2014-spring-mlc-sol.pdf}{solutionnaire}), ainsi que \href{http://www.soa.org/files/edu/edu-2014-spring-mlc-ques-sol.pdf}{22 questions types à développement} (les questions et leurs solutions sont comprises dans le document). \vspace{\baselineskip}

Au sujet des manuels d'étude, le plus populaire au sein des étudiants du baccalauréat semble être le manuel ASM, rédigé par Abraham Weishaus. Les commentaires sur ce manuel de 1927 pages sont assez bons, et la majorité des gens s'entendent pour dire qu'il couvre bien toute la matière du syllabus de l'examen. Par contre, certaines personnes ont comme avis que le manuel ASM passe trop rapidement sur certains concepts, ou bien présente parfois des raccourcis, voire même du par coeur, sans miser totalement sur la compréhension globale de la matière. Ceux qui préfèreraient apprendre avec une compréhension plus globale de la matière et qui ne voient pas de problème à dépenser un certain montant peuvent se tourner vers le cours en ligne de \href{http://www.theinfiniteactuary.com/exams/45}{\textit{The Infinite Actuary}} ou celui de \href{https://www.coachingactuaries.com/ltam/}{\textit{Coaching Actuaries}}. Ces cours présentent la matière en vidéos d'une manière que plusieurs trouvent dynamique et motivante. Ces cours en ligne possèdent aussi des solutions à des centaines d'exercices ainsi que plusieurs examens pratiques. \vspace{\baselineskip}

Les ressources les plus populaires demeurent néanmoins le manuel ASM accompagné des questions types et des anciens examens de la SOA. Un conseil qui revient souvent de ceux qui ont fait l'examen est de ne pas négliger les questions à développement, car c'est souvent là que les personnes moins préparées manquent de temps à l'examen. \vspace{\baselineskip}

Pour ce qui est d'ADAPT, certains trouvent que c'est une ressource utile pour cet examen, mais d'autres affirment qu'ADAPT reprend beaucoup de questions déjà publiées par la SOA, que ce soient des questions types ou des questions tirées des anciens examens. Certains estiment donc qu'ADAPT n'est pas un investissement nécessaire pour l'examen LTAM si l'on a les questions types et les anciens examens en notre possession, mais le choix reste le vôtre. \vspace{\baselineskip}

\textbf{Changements du curriculum de juillet 2018:} Certaines notions de l'ancien examen MLC sont retirées, comme les courbes de rendement et les notions portant sur le risque diversifiable. C'est aussi le cas de l'assurance vie universelle, qui est maintenant vue lors des modules FAP. Également, certains sujets qui étaient originellement au syllabus de l'examen C sont maintenant évalués dans cet examen, de même que des notions supplémentaires sur les produits d'assurance vie et de rentes. Je vous conseille de consulter le syllabus pour plus d'informations sur les notions traitées à l'examen. \vspace{\baselineskip} \newpage




\subsection*{Examen SRM}
\label{subsec:examsrm}
\addcontentsline{toc}{subsection}{Examen SRM}

\href{https://www.soa.org/Education/Exam-Req/edu-exam-srm-detail.aspx}{L'examen SRM} (SRM pour \textit{Statistics for Risk Modeling}) est un examen totalement nouveau du côté de la SOA. Cet examen a été instauré lors des changements aux prérequis pour devenir Associé de juillet 2018. La SOA, en sondant les employeurs et les institutions académiques, a conclu que les aspirants actuaires devaient démontrer des compétences en analyse prédictive qui vont plus loin que les simples notions académiques de régression linéaire et de séries chronologiques qui étaient testées avant juillet 2018. Il a été décidé que ces nouvelles compétences en analyse prédictive seraient d'abord évaluées de manière classique lors de l'examen SRM, puis de manière plus pratique dans \href{https://www.soa.org/Education/Exam-Req/edu-exam-pa-detail.aspx}{l'examen PA} (voir la section suivante). Il est également à noter que les connaissances du VEE \textit{Mathematical Statistics} sont jugées comme prérequises à cet examen. \vspace{\baselineskip}

L'examen SRM teste des connaissances en modèles linéaires, en séries chronologiques, en arbres de décision, en partitionnement de données et en analyse en composantes principales. Les candidats doivent également être en mesure de choisir et de valider des modèles statistiques. Les cours du baccalauréat en lien avec cet examen sont, entre autres, \textit{ACT-2003 Modèles linéaires en actuariat} et \textit{ACT-2010 Séries chronologiques}. Pour consulter les objectifs spécifiques de l'examen ainsi que d'autres informations, voir le syllabus. Cet examen est d'une durée de 3 heures 30 et comporte 35 questions à choix multiples. La première séance de l'examen SRM se tiendra en septembre 2018. \vspace{\baselineskip}

Pour l'instant, il y a encore peu de ressources disponibles pour cet examen. La SOA a toutefois publié \href{https://www.soa.org/Files/Edu/2018/exam-srm-sample-questions.pdf}{un document} comprenant 28 questions types et leurs solutions. Des ressources additionnelles devraient probablement être publiées dans le futur. \vspace{\baselineskip} \newpage


\subsection*{Examen PA}
\label{subsec:exampa}
\addcontentsline{toc}{subsection}{Examen PA}

\href{https://www.soa.org/Education/Exam-Req/edu-exam-pa-detail.aspx}{L'examen PA} (PA pour \textit{Predictive Analytics}) est également un examen totalement nouveau du côté de la SOA. Cet examen a été instauré lors des changements aux prérequis pour devenir Associé de juillet 2018. Il se démarque des autres examens en ce qu'il n'utilise pas la même méthode que les examens précédents pour tester la connaissance d'objectifs d'apprentissage. Maintenant que les nouvelles compétences en analyse prédictive ont été testées de manière classique lors de \href{https://www.soa.org/Education/Exam-Req/edu-exam-srm-detail.aspx}{l'examen SRM} (voir la section précédente), elles le sont maintenant de manière plus pratique dans cet examen PA. Il est donc obligatoire d'avoir fait et réussi l'examen SRM (ou bien d'avoir un crédit de transition pour cet examen) avant de pouvoir s'inscrire à l'examen PA. \vspace{\baselineskip}

On peut affirmer que l'examen PA est séparé en 2 parties:
\begin{enumerate}
\item La première partie consiste en des modules d'apprentissage en ligne qui ont pour seul but de préparer à la deuxième partie. Ces modules poussent plus loin les notions apprises pour l'examen SRM et clarifient les attentes de la SOA au sujet de la deuxième partie.
\item La deuxième partie consiste en la réalisation pratique d'un rapport. Les candidats se font présenter un jeu de données et un problème corporatif jugé réaliste. Les candidats ont 5 heures pour préparer un rapport qui présente leur solution à ce problème.
\end{enumerate}
\vspace{\baselineskip}

L'examen se déroule dans un centre Prometric, et les candidats travaillent sur un ordinateur où ils peuvent utiliser les programmes Excel, Word et RStudio. À la fin des 5 heures, les candidats doivent remettre un rapport et tout le code R et les autres éléments qui peuvent supporter la solution qu'ils ont présentée dans leur rapport. La première séance de l'examen PA se tiendra en décembre 2018. De par sa nature, l'examen ne sera pas corrigé comme les examens préliminaires qui le précèdent, mais plutôt selon \href{https://www.soa.org/Education/General-Info/edu-guide-written-exams-seminar-vids.aspx}{la méthode utilisée pour corriger les examens avancés de Fellowship}. \vspace{\baselineskip}

Tout comme pour l'examen SRM, il y a encore très peu de ressources disponibles au sujet de l'examen PA. La SOA prévoit toutefois produire un exemple de problème et de rapport d'ici la première séance d'examen en décembre 2018 pour aider la préparation des candidats. \vspace{\baselineskip}

\newpage

\section*{Les examens préliminaires (partie II) - Côté CAS}
\label{sec:prelimssuitecas}
\addcontentsline{toc}{section}{Les examens préliminaires (partie II) - Côté CAS}

Cette section est encore en développement. \vspace{\baselineskip}


\subsection*{Examen MAS-I (Anciennement S)}
\label{subsec:examMAS_I}
\addcontentsline{toc}{subsection}{Examen MAS-I (Anciennement S)}

\href{http://www.casact.org/admissions/syllabus/index.cfm?fa=MASI&parentID=391}{L'examen MAS-I} (MAS-I pour \textit{Modern Actuarial Statistics-I}) est une introduction aux modèles probabilistes (les processus stochastiques et les modèles de survie), à l'utilisation des tests statistiques, aux modèles linéaires et aux séries chronologiques. Les cours du baccalauréat en lien avec cet examen sont \textit{ACT-2009 Processus stochastiques}, \textit{ACT-2003 Modèles linéaires en actuariat} et \textit{ACT-2010 Séries chronologiques}. Pour consulter les objectifs spécifiques de l'examen ainsi que d'autres informations, voir le syllabus. Cet examen est d'une durée de 4 heures et devrait comporter 40 questions à choix multiples à correction négative comme l'ancien examen S. La CAS a publié des anciens \href{http://www.casact.org/admissions/studytools/examS/}{examens} types du S, ainsi que quelques \href{http://www.casact.org/admissions/studytools/examS/Sample_Questions.pdf}{questions} de l'examen S.\vspace{\baselineskip}

Pour ce qui est du manuel d'étude à utiliser, il y a le manuel d'\href{https://drive.google.com/open?id=0B6kXivc6X9LISGhkVUkzLW5sSnc}{ASM}, encore une fois écrit par Abraham Weishaus. Par contre, certaines personnes estiment que les \href{https://drive.google.com/open?id=0B6kXivc6X9LIOUs3SDF3NmVKNGM}{examens pratiques} d'Howard Mahler sont très similaires aux examens et donc plus adéquats pour l'étude. Par contre, avec les changements, la situation pourrait changer. \vspace{\baselineskip}


\subsection*{Examen MAS-II (Anciennement C)}
\label{subsec:examMAS_II}
\addcontentsline{toc}{subsection}{Examen MAS-II (Anciennement C)}

\href{http://www.casact.org/admissions/syllabus/index.cfm?fa=MASII&parentID=392}{L'examen MAS-II} (MAS-II pour \textit{Modern Actuarial Statistics-II}) est une introduction à la modélisation, à l'analyse de données statistiques dans un contexte d'affaires, à la crédibilité, à l'estimation et à la création d'intervalles de confiance pour les décisions basées sur les modèles statistiques. Les cours du baccalauréat en lien avec cet examen sont \textit{ACT-2001 Introduction à l'actuariat II}, \textit{ACT-2005 Mathématiques actuarielles IARD I} et \textit{ACT-2008 Mathématiques actuarielles IARD II}. Pour consulter les objectifs spécifiques de l'examen ainsi que d'autres informations, voir le syllabus. Cet examen est d'une durée de 4h00 à correction négative et le nombre de questions n'est pas encore annoncé. La SOA a publié \href{http://www.soa.org/files/edu/edu-exam-c-sample-quest.pdf}{305 questions} types pour cet examen, accompagnées du \href{http://www.soa.org/files/edu/edu-exam-c-sample-sol.pdf}{solutionnaire}. Il s'agit du matériel d'étude du C, du nouveau matériel d'étude sera disponible d'ici là.\vspace{\baselineskip}

Pour ce qui est du manuel d'étude et des examens formatifs à utiliser, le matériel devrait être similaire à l'ancien C (soit le manuel de ASM, encore une fois écrit par Abraham Weishaus et le \href{http://howardmahler.com/Teaching/C.html}{manuel} d'Howard Mahler). Plus d'information sera ajoutée avec les développements des changements.\vspace{\baselineskip}

\subsection*{Examen 5}
\label{subsec:exam5}
\addcontentsline{toc}{subsection}{Examen 5}

Sous-section en développement. \vspace{\baselineskip}


\subsection*{Examen 6}
\label{subsec:exam6}
\addcontentsline{toc}{subsection}{Examen 6}

Sous-section en développement. \vspace{\baselineskip}


\newpage
